%% start of file `template.tex'.
%% Copyright 2006-2013 Xavier Danaux (xdanaux@gmail.com).
%
% This work may be distributed and/or modified under the
% conditions of the LaTeX Project Public License version 1.3c,
% available at http://www.latex-project.org/lppl/.
%Version for spanish users, by dgarhdez

\documentclass[11pt,a4paper,roman]{moderncv}        % possible options include font size ('10pt', '11pt' and '12pt'), paper size ('a4paper', 'letterpaper', 'a5paper', 'legalpaper', 'executivepaper' and 'landscape') and font family ('sans' and 'roman')
\usepackage[spanish,es-lcroman]{babel}


% moderncv themes
\moderncvstyle{classic}                            % style options are 'casual' (default), 'classic', 'oldstyle' and 'banking'
\moderncvcolor{green}                              % color options 'blue' (default), 'orange', 'green', 'red', 'purple', 'grey' and 'black'
%\renewcommand{\familydefault}{\sfdefault}         % to set the default font; use '\sfdefault' for the default sans serif font, '\rmdefault' for the default roman one, or any tex font name
%\nopagenumbers{}                                  % uncomment to suppress automatic page numbering for CVs longer than one page

% character encoding
\usepackage[utf8]{inputenc}                       % if you are not using xelatex ou lualatex, replace by the encoding you are using
%\usepackage{CJKutf8}                              % if you need to use CJK to typeset your resume in Chinese, Japanese or Korean

% adjust the page margins
\usepackage[scale=0.75]{geometry}
%\setlength{\hintscolumnwidth}{3cm}                % if you want to change the width of the column with the dates
%\setlength{\makecvtitlenamewidth}{10cm}           % for the 'classic' style, if you want to force the width allocated to your name and avoid line breaks. be careful though, the length is normally calculated to avoid any overlap with your personal info; use this at your own typographical risks...

% personal data
\name{Roy}{Luo}
\title{Resumé title}                               % optional, remove / comment the line if not wanted
\address{Markham, Ontario}% optional, remove / comment the line if not wanted; the "postcode city" and and "country" arguments can be omitted or provided empty
\phone[mobile]{4373634171}                   % optional, remove / comment the line if not wanted
%\phone[fixed]{+2~(345)~678~901}                    % optional, remove / comment the line if not wanted
%\phone[fax]{+3~(456)~789~012}                      % optional, remove / comment the line if not wanted
\email{royluocanada@gmail.com}
\date{September 13, 2024}                              % optional, remove / comment the line if not wanted
%\homepage{www.johndoe.com}                         % optional, remove / comment the line if not wanted
%\extrainfo{additional information}                 % optional, remove / comment the line if not wanted
%\photo[64pt][0.4pt]{picture}                       % optional, remove / comment the line if not wanted; '64pt' is the height the picture must be resized to, 0.4pt is the thickness of the frame around it (put it to 0pt for no frame) and 'picture' is the name of the picture file
%\quote{Some quote}                                 % optional, remove / comment the line if not wanted

% to show numerical labels in the bibliography (default is to show no labels); only useful if you make citations in your resume
%\makeatletter
%\renewcommand*{\bibliographyitemlabel}{\@biblabel{\arabic{enumiv}}}
%\makeatother
%\renewcommand*{\bibliographyitemlabel}{[\arabic{enumiv}]}% CONSIDER REPLACING THE ABOVE BY THIS

% bibliography with mutiple entries
%\usepackage{multibib}
%\newcites{book,misc}{{Books},{Others}}
%----------------------------------------------------------------------------------
%            content
%----------------------------------------------------------------------------------
\begin{document}
%-----       letter       ---------------------------------------------------------
% recipient data
\recipient{\phantom{1}}{}
\opening{\phantom{1}}
\closing{\phantom{1}}
\enclosure[\phantom{1}]{\phantom{1}}          % use an optional argument to use a string other than "Enclosure", or redefine \enclname
\makelettertitle

Dear Hiring Manager,

I am excited to apply for the Software Engineering Internship at Stripe, as the opportunity to contribute to impactful projects while learning from experienced engineers aligns perfectly with my career aspirations. As a third-year statistics student, I have gained a strong foundation in computer science fundamentals, including data structures, algorithms, and operating systems. My coursework, combined with personal coding projects, has honed my ability to tackle complex problems and deliver clean, well-documented code.

I am particularly drawn to the prospect of working at Stripe due to the innovative work being done in engineering and the emphasis on ownership and collaboration. I am confident that my proficiency in programming languages such as Python along with my passion for developing scalable data solutions, will allow me to make meaningful contributions during the internship.

Outside of academics, I have gained valuable experience through participating in hackathons and working on personal coding projects, where I applied problem-solving skills in real-world scenarios. These experiences have strengthened my technical capabilities and fueled my curiosity to learn more. I am eager to bring this drive to Stripe, where I look forward to growing professionally, collaborating with a diverse team, and contributing to the company's data infrastructure. 

I am excited about the chance to join Stripe for the summer internship, and I am confident that my skills, passion for learning, and strong work ethic will allow me to make a positive impact.

Sincerely,

Roy

\vspace{0.5cm}

\end{document}


%% end of file `template.tex'.
